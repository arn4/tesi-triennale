\documentclass[a4]{article}
\usepackage{pack}

\geometry{
	twocolumn,
	a4paper,
	total={170mm,257mm},
	left=20mm,
	top=20mm,
}
\title{Note e riflessioni per la tesi}
\author{Luca Arnaboldi}

\begin{document}
\maketitle
\tableofcontents
Il setting è quello degli \emph{open quantum systems}. In particolare vogliamo studiare
la markovianità dell'evoluzione temporale, spesso chiamata \emph{divisibilità}, anche se
le due definizioni non coincidono. Lo scopo della tesi è indigare le differenze tra questi
due concetti. Il lavoro si basa sul contenuto dell'articolo \emph{Complete Positive
Divisibilitydoes not mean Markovianity} di Modi~et.~al.\cite{CPdoesnotimply}.

\section{Notazione}
Descriveremo un sistema \(S\), a contatto con un ambiente \(E\); i rispettivi spazi di
Hilbert sono \(\Hilb_S\) e \(\Hilb_E\), supposti tutti di dimensione finita. Lo spazio
su cui si lavora è dunque 
\(\Hilb_{SE} = \Hilb_{SE} \otimes \Hilb_{SE}\), e la più generica Hamiltoniana si
scriverà come
\[\hat{H}_{SE} = \hat{H}_S \otimes  \Id_E + \Id_S \otimes  \hat{H}_E + \hat{V}_{SE}.\]

L'insieme degli operatori lineari di \(\Hilb\) è \(\LinSet(\Hilb)\), mentre quello delle
matrici densità è \(\sigma(\Hilb)\); ovviamente \(\LinSet(\Hilb) \subset \sigma(\Hilb)\).
Indicheremo con \(\hat{\rho}_S\) e \(\hat{\tau}_E\) due generiche matrici densità di \(S\)
e \(E\) rispettivamente; la matrice di densità completa è invece indicata da \(\hat{\rho}_{SE}\).
Ovviamente tutte dipendendono dal tempo.


Indicheremo con \(\Phi_{t'\to t}(\cdot)\) la mappa che evolve che manda \(\hat{\rho}_S(t')\)
in \(\hat{\rho}_S(t)\). Se omettiamo \(t'\) significa che sottointendiamo che parte da \(t'=0\).
Per brevità, a volte indicheremo \(\Phi_{t'\to t}\) con \(\Phi_{t:t'}\) (attenzione il verso è scambiato!). Infine se la mappa \(\Phi\) è accessibile sperimentalmente verrà indicata con
\(\Lambda\).

\section{Definizion \& fatti utili: ripasso}
Qui metto un po' di cose ``scolastiche'', cioè prese dai corsi di \texttt{MQ} o \texttt{QINFO}
che però ci tengo a ripassare per la tesi.
\subsection{Misure \& POVM}
Ho preso tutto dal Nielsen \cite{nielsen2010quantum}.

Il postulato di misura della meccanica quantistica (\emph{misure proiettive}) è particolarmente
restrittivo. Ad esempio non contempla casi in cui la misura è irripetibile: se ripeto una
seconda volta una misura proiettiva non perturbo nuovamete il sistema ed ottengo lo stesso
risultato. Chiaramente però questo tipo di misura non può ovviamente descrivere il caso in
cui, ad esempio, ho misurato un fotone distuggendolo.
\begin{post}[di misura]
Le misure in meccanica quantistica sono descritte da un insieme di operatori
\(\{\hat{M}_m\}\), tali che rispettano l'equazione di completezza
\begin{equation} \label{eq:compmis}
\sum_m \hat{M}^\dag_m \hat{M}_m = \Id
\end{equation}
e l'indice \(m\) si riferisce al risultato della misura.

Se il sistema si trova nello stato \(\ket{\psi}\) prima della misura, si troverà successivamente
nello stato
\[ \frac{\hat{M}_m \ket{\psi}}{\sqrt{\Braket{\psi|\hat{M}^\dag_m \hat{M}_m|\psi}}}\]
con probabilità \(p(m)= \braket{\psi|\hat{M}^\dag_m \hat{M}_m|\psi}\).
\end{post}
L'equazione \eqref{eq:compmis} discende dalla richiesta che le probabilità sommino a 1:
\(\sum_m~p(m)~=~1\).

I POVM sono utili quando non siamo interessati allo stato del sistema dopo la misura.
\begin{defn}[POVM]
Un \emph{POVM} è un insieme di operatori \emph{positivi} \(\{\hat{E}_m\}\) tali che
\[\sum_m \hat{E}_m  = \Id.\]
\end{defn}
È una buona definizone perchè scegliendo \(\hat{M}_m = \sqrt{\hat{E}_m}\) otteniamo una
misura valida per il postulato di misura. I POVM forniscono infomazioni \emph{solo sulla
probabilità di misura} e non sullo stato risultante siccome comunque scegliendo
\[\hat{M}_m = \hat{U}\sqrt{\hat{E}_m},\]
dove \(\hat{U}\) è una qualsiasi unitaria, si ottiene una misura con la stessa statistica
del POVM, ma che ovviamente non ha lo stesso stato finale. Questa ambiguità
rende i POVM inutilizzabili nei casi in cui è necessario conoscere lo stato.

Una misura proiettiva si ottiene scegliendo come come operatori di misura
i poiettori. Analogamente la descrizione POVM di una misura poiettiva è
\[\hat{E}_m = \ket{m}\bra{m}.\]

Per concludere, la probabilità di ottenere il risultato \(m\) da una misura su
una matrice di densità è
\[p(m) = \Tr[\hat{E}_m \hat{\rho}]\]



\subsection{Open Quantum Systems}
Per le note proprietà delle matrici di densità vale 
\[\hat{\rho}_S(t) = \Tr_E\left(\hat{\rho}_{SE}(t)\right).\]
Ovviamente il sistema congiunto \(S-E\) evolve come un sistema chiuso, quindi attraverso un
operatore unitario \(\hat{U}_{SE}(t, t')\):
\[\hat{\rho}_{SE}(t) = \hat{U}_{SE}(t, t') \hat{\rho}_{SE}(t') \hat{U}^{\dag}_{SE}(t, t'),\]
ma lo stesso non vale per il sistema \(S\) da solo. Infatti lui non è chiuso, dato che interagisce
con l'ambiente; \(S\) è apputno detto \emph{open quantum system}.

\begin{defn}[Completamente Positivo] Una mappa \(\Phi\colon \sigma(\Hilb) \to \sigma(\Hilb) \)
si dice \emph{completamente postiva} se dato un qualsiasi sistema \(A\), la mappa
\[\Phi \otimes \Id_A: \sigma(\Hilb \otimes \Hilb_A) \to \sigma(\Hilb \otimes \Hilb_A) \]
è positiva.
\end{defn}

\(\Phi_{t'\to t}(\cdot)\) è lineare, completamente positiva e conserva la traccia,
più brevemente  diremo \emph{LCPT}. CP è vero nel caso in cui lo stato iniziale di
S-E è fattorizzabile (vedi tipo \cite{breuer2007theory} per i dettagli).

\begin{defn}[Divisibilità completamente positiva] Un sistema è \emph{CP-divisibile} se la sua
mappa \(\Phi\) è tale che
\begin{equation} \label{eq:CPdiv}
\Phi_{t'\to t} = \Phi_{s\to t} \circ \Phi_{t'\to s} \quad \forall t, t', s: t' \le s \le t.
\end{equation}
\end{defn}
Al corso di \texttt{QINFO2} abbiamo detto che 
\[\text{markoviano quantistico} \iff \text{CP-divisibile}, \]
e che un criterio sufficiente è che la distanza traccia sia decrescente. Queste cose in realtà
sono entrambe false \cite{markovcondition}, e quindi c'è del lavoro da fare.

\section{Inquadriamo il problema}
La definizione classica di processo markoviano è ben nota (vedi tipo \cite{breuer2007theory, markovcondition}).
Il problema se condideriamo processi quantistici è che, nonstante sia matematicamente
ben definita, \emph{operativamente} è inutilizzabile: non posso infatti verificare
la markovianità (che è una condizione ad  ogni tempo) senza perturbare il sistema
quando misuro. In ultima analisi quindi  la definizione classica di markovianità
\textbf{non} è utilizzabile nel linguaggio quantistico.
La gente ha fatto diversi tentativi per risolvere il problema, ma quasi tutti seguono
il pattern: prendo una condizione necessaria, \emph{ma non sufficiente} alla markovianità
classica, e la estendo facendola diventare una condizione quantistica. La CP-divisibilità
è uno di questi tentativi: ho preso l'equazione di Chapman-Kolmogorov (che è in un certo
senso la proprietà più importante dei processi markoviani), e l'ho trasformata in una
codizione  quantistica.
È stata  trovata una buona definizione di markovianità quantistica, che concide con
quella classica per il limite giusto, è tutto esposto nel \cite{markovcondition}.

Il problema è dunque capire come si inquadra in questo  contesto la completa positività.
I problemi sono molteplici: l'attuale definizione di CP-positività non si presta alla
verifica in laboratorio  (nel senso  che manca una definizione precisa su che tomografia
quantistica effettuare per verificare la completà positività), dovremo dunque occuparci
di formularla in una definizione operativa.
Infine vedremo come metterla in relazione con la markovianità quantistica.

\section{Contenuto}
Esponiamo innanzi tutto due defiizioi non equivalenti della \emph{CP-sivisibilità}.

\paragraph{ICP divisibilità}
Siamo in un contesto sperimentale dove possiamo preparare un sistema in un qualsiasi
stato iniziale al tempo \(r=0\), e poi effetuare misure a qualsiasi tempo \(s>0\).
Con la tomografia  quantistica  possiamo costruire  l'insieme
\[  \lambda_0 = \{\Lambda_{s\colon0}\}, \]
delle mappe che descrivono la dinamica. Supponiamo che \textbf{tutte le mappe di \(\boldmath{\lambda_0}\) siano invertibili.}
\begin{defn}[ICP divisibilità]
Un processo è \emph{CP-divisibile per inversione} se, dati \(s \le  t\), la mappa
\[\Phi_{t\colon s} = \Lambda_{t\colon0} \circ \Lambda^{-1}_{s\colon0}\]
è completamente positiva.
\end{defn}
Ovviamente se un processo è ICPdiv allora ogni mappa \(\Phi_{t\colon s}\) è ben definita
a partire da \(\lambda_0\).

\paragraph{OCP divisibilità}
Supponiamo invece che di poter manipolare il sistema a qualsiasi tempo \(s\). In
particolare possiamo interrompere l'evoluzione ad un istante \(s_-\), effettuare una
misura, e sostituire lo stato con una qualsiasi altra matrice di densità in modo che
l'evoluzione continui dall'istante \(s_+\). In questa situazione è  possibile  ricostruire
sperimentalmente tutte le mappe dell'insieme
\[\lambda = \{\Lambda_{t\colon s}\}  \quad \forall s,t\colon s \le t.\]  
\begin{defn}[OCP  divisibilità]
Un processo è \emph{CP-divisibile operativamente} se dati \(r \le s \le  t\), vale la
relazione 
\begin{equation} \label{eq:OCPdiv}
\Lambda_{t\colon r} = \Lambda_{t\colon s} \circ \Lambda_{s\colon r},
\end{equation}
dove le mappe appartengono tutte all'insieme \(\lambda\).
\end{defn}
Le equazioni \eqref{eq:CPdiv} e \eqref{eq:OCPdiv} sono formalmente identiche, con la differenza che nella seconda è chiaro come verificarla operativamente.



\printbibliography


\end{document}