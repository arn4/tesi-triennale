
Isolare completamente un sistema quantistico è impossibile, si introduce di conseguenza la nozione di \emph{open quantum system}, per separare la descrizione del sistema \(S\), accessibile sperimentalmente, dall'ambiente \(E\) con cui può interagire. L'insieme di \(S\) ed \(E\) è un sistema chiuso, dunque evolverà sotto l'azione dell'operatore unitario \(\hat{U}_{SE}(t,s)\); conseguentemente uno stato \(\hat{\rho}_s\) del sistema \(S\) evolverà nello stato
\[\Phi_{t\colon s}\left[\hat{\rho}_s\right]=\hat{\rho}_t = \Tr_E\left(\hat{U}^\dag_{SE}(t,s)(\hat{\rho}_s\otimes\eta_s)\hat{U}_{SE}(t,s)\right),\]
dove \(\eta_s\) è lo stato dell'ambiente quando inizia l'evoluzione, ed è stata introdotta la mappa \(\Phi_{t\colon s}\) che descrive l'evoluzione del sistema \(S\) tra i tempi \(s\) e \(t\). Si verifica che \(\Phi_{t\colon s}\) è lineare, preserva la traccia e completamente positiva.

In quest'ambito è di interesse lo studio della \emph{markovianità} del processo, ovvero l'assenza di effetti di memoria, introdotti dall'ambiente. Nonostante la markovianità classica sia ben formulata e definita, la ricerca di una condizione che la caratterizzi completamente a livello quantistico è ancora in corso. In questo elaborato si espone la soluzione proposta da Modi~et~al \cite{CPdoesnotimply}, dove si mostra che la \emph{CP-divisibilità}, spesso usata come discriminante per processi markoviani, non è in realtà sufficiente a caratterizzare tutti gli effetti di memoria che possono essere presenti in un processo quantistico. Nell'articolo si propongono 2 definizioni differenti di divisibilità, ponendo attenzione sul loro significato sperimentale; successivamente vengono discusse quali relazioni ci sono tra queste definizioni, e come esse si rapportano alla markovianità.

Supponiamo di disporre di un apparato sperimentale che possa preparare il sistema in uno stato qualsiasi \(\hat{\rho}_r\) ad un tempo iniziale \(r\). Indichiamo con \(\Lambda\) le mappe completamente positive accessibili sperimentalmente. Supponiamo inoltre di poter effettuare misure sul sistema \(S\) ad un qualsiasi tempo \(s>t\). In tale modo possiamo caratterizzare tutte le mappe della famiglia \(\lambda_0 := \{\Lambda_{s\colon r}\}\), che regolano l'evoluzione del sistema \(S\). Diremo che un processo è \emph{CP divisibile per inversione (iCP)} se tutte le mappe di \(\lambda_0\) sono invertibili, e \[\Phi_{t\colon s} = \Lambda_{t\colon r} \circ \Lambda^{-1}_{s\colon r}\] è completamente positiva \(\forall r<s<t \). Se invece l'apparato permette anche la preparazione del sistema in un nuovo stato \(\hat{\rho}_s\) istantaneamente dopo aver effettuato la misura ad \(s\), allora le mappe accessibili sperimentalmente sono \(\lambda = \{\Lambda_{t\colon s}\}\), e diremo che il processo è \emph{CP divisibile operativamente (oCP)} se vale
\[\Lambda_{t\colon r} = \Lambda_{t\colon s} \circ \Lambda^{-1}_{s\colon r} \quad \forall r < s <  t.\]
Le due definizioni non sono ovviamente equivalenti, dato che la prima richiede necessariamente che l'invertibilità delle mappe di \(\lambda_0\), mentre la divisibilità operazionale non ha bisogno di questa condizione. Tuttavia, anche limitandosi al caso in cui effettivamente si ha invertibilità, le due definizioni differiscono: si può mostrare che \emph{oCP} implica \emph{iCP}, ma esistono controesempi che smentiscono il viceversa. Infine, si può mostrare che nonostante sia una condizione più stringente, l'oCP divisibilità non caratterizza completamente la markovianità, dato che si possono trovare ancore alcuni effetti di memoria. L'oCP divisibilità corrisponde a quella che si può chiamare \emph{markovianità in media}.
