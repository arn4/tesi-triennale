\documentclass{beamer}
\usepackage{pack}

\usepackage[utf8]{inputenc}

\usetheme{Madrid}
\usecolortheme{seahorse} %seahorse, beaver

\setitemize{label=\usebeamerfont*{itemize item}%
	\usebeamercolor[fg]{itemize item}
	\usebeamertemplate{itemize item}}

\makeatletter
\g@addto@macro\normalsize{%
	\setlength\belowdisplayskip{-0pt}
}

%Information to be included in the title page:
\title[Caratterizzazione MD in regime quantistico]
      {Caratterizzazione di dinamiche markoviane in regime quantistico}
%\subtitle[it]{\texttt{Inserisci sottotitolo}}
\author[Luca Arnaboldi]{Luca Arnaboldi \\[1ex]
	{\small Relatore: Vittorio Giovannetti}}      
\institute[]{\includegraphics[height=2.60cm]{img/unipi.png}}
\date[17-09-2020]{17 settembre 2020}
%\logo{\includegraphics[height=1.45cm]{img/unipi-noscritta.png}}

\begin{document}
	
\frame[plain]{\titlepage}

% Slide 1
\begin{frame}
	\frametitle{Open Quantum Systems \& Markovianità}
	\begin{columns}
		\begin{column}{0.73\textwidth}
			Sistema \(S\) che interagisce con l'ambiente esterno \(E\).
			\begin{block}{}
				\(S+E\) è un chiuso \(\implies\) evoluzione~unitaria~\(\hat{U}_{SE}\)
			\end{block}
		    \vspace*{-\baselineskip}\setlength\belowdisplayshortskip{0pt}
			\[\Phi_{t\colon s}\left[\hat{\rho}_s\right]=\hat{\rho}_t = \Tr_E\left(\hat{U}_{SE}(t,s)(\hat{\rho}_s\otimes\hat{\eta}_s)\hat{U}^\dag_{SE}(t,s)\right)\]
		\end{column}
	    \begin{column}{0.22\textwidth}
	    	\begin{figure}
	    		\resizebox{\textwidth}{!}{\begin{tikzpicture}
\draw[rounded corners, dashed] (0.75,0.75) rectangle (4.75,4.77);
\filldraw[rounded corners, fill=pastelgreen] (1,1) rectangle (4.5,2);
\draw[double arrow=3pt colored by black and pastelred]
    (2.5,2.05) arc(210:150:.25) arc(-30:30:.25) arc(210:150:.25) arc(-30:30:.25)-- ++(90:.5);
\draw[double arrow=3pt colored by black and pastelred]
    (3,3.50) arc(150:210:.25) arc(30:-30:.25) arc(150:210:.25) arc(30:-30:.25)-- ++(270:.5);
\filldraw[rounded corners, fill=pastelblue] (2.25,3.52) rectangle (3.25,4.52);

\node[white] at (2.75,4.02) {\(S\)};
\node[white] at (2.75,1.5) {Ambiente \(E\)};
%    \filldraw[rounded corners, fill=pastelyellow] (0,0) rectangle (6,6);
%    \filldraw[fill=pastelgreen] (3,3) circle (1.3cm);
%    %E
%    \draw[double arrow=3pt colored by black and pastelred]
%    (4.4,3) arc(120:60:.25) arc(-120:-60:.25)arc(120:60:.25) arc(-120:-60:.25) -- ++(0:.5);
%    %N
%    \draw[double arrow=3pt colored by black and pastelred]
%    (3,4.4) arc(210:150:.25) arc(-30:30:.25) arc(210:150:.25) arc(-30:30:.25)-- ++(90:.5);
%    %W
%    \draw[double arrow=3pt colored by black and pastelred]
%    (1.6,3) arc(60:120:.25) arc(-60:-120:.25)arc(60:120:.25) arc(-60:-120:.25) -- ++(180:.5);
%    %S
%    \draw[double arrow=3pt colored by black and pastelred]
%    (3,1.6) arc(150:210:.25) arc(30:-30:.25) arc(150:210:.25) arc(30:-30:.25)-- ++(270:.5);
    
\end{tikzpicture}}
	    	\end{figure}
	    \end{column}
	\end{columns}
    \pause
    \begin{block}{Markovianità classica}
    	\(\mathbb{P}\left(X_{n+1} = a_{n+1} | X_{n} = a_{n}, \dots, X_0 = a_0\right) =
    	\mathbb{P}\left(X_{n+1} = a_{n+1} | X_{n} = a_{n}\right)
    	%\quad \forall n\in \mathbb{N}.
    	\)
    \end{block}
    \pause
    \begin{columns}
    	\begin{column}{0.07\textwidth}
    		\(\Phi\)~è
    	\end{column}
    	\begin{column}{0.30\textwidth}
    		\begin{itemize}
    			\item[L] lineare
    			\item[CP] completamente positiva
    			\item[T] preserva la traccia
    		\end{itemize}
    	\end{column}
    	\pause
    	\begin{column}{0.56\textwidth}
    		\begin{alertblock}{Markovianità quantistica?}
    			Problema aperto.
    			\textbf{Non} è la CP~divisibilità
    			\vspace*{-\baselineskip}
    			\begin{equation*}
    				\Phi_{t\colon r} = \Phi_{t\colon s} \circ \Phi_{s\colon r}
    		    \end{equation*}
    	        con \(\Phi_{x\colon y}\) CP \(\forall x,y\)
    	    \end{alertblock}
    	\end{column}
    \end{columns}


    
\end{frame}

% Slide 2
\begin{frame}
\frametitle{Divisibilità iCP}
Trattazione di \emph{Modi et al} \cite{CPdoesnotimply}: definizione \emph{operative} della divisibilità e rapporti tra esse.

\(\Lambda \): canale quantistico accessibile sperimentalmente.
\pause
\begin{block}{CP divisibilità per inversione [iCP]}
\begin{columns}
	\begin{column}{0.5\textwidth}
		\begin{itemize}
			\item[r] preparazione stato \(\hat{\rho}_r\)
			\item[s] misurazione dello stato \(\hat{\rho}_s\)
		\end{itemize}
	\end{column}
	\begin{column}{0.3\textwidth}
		\resizebox{\textwidth}{!}{
\begin{tikzpicture}
\filldraw[rounded corners, fill=pastelblue] (-1,0) -- (1,0) -- (0,-1) -- cycle;
\filldraw[rounded corners, fill=pastelgreen] (-0.5,3) -- (-0.5,1) -- (-1.5,2) -- cycle;
\draw [rounded corners,line width=1mm, black] (0,0) -- (0,1) -- (1.5,1);
\draw [line width=1mm, black] (-0.5,2) -- (1.5,2);
\filldraw[rounded corners, fill=pastelred] (1.5,0.5) rectangle (3,2.5);
\draw [line width=1mm, black] (3,2) -- (3.5,2);
\draw [rounded corners,line width=1mm, black] (3.2,1.7) -- (3.8,2.3);
\draw [line width=1mm, black] (3.0,1) -- (3.3,1);

\node[white] at (-0.9,2) {\LARGE \(\hat{\eta}_r\)};
\node[white] at (0,-0.4) {\LARGE \(\hat{\rho}_r\)};
\node at (3.6,1) {\LARGE \(\hat{\rho}_{t}\)};

\node[white] at (2.3,1.5) {\LARGE \(\hat{U}_{t\colon r}\)};
\end{tikzpicture}
}
	\end{column}	
\end{columns}
 Canali accessibili sperimentalmente: \(\lambda_0 := \{\Lambda_{s\colon r}\}\). Il sistema è \emph{iCP} se:
 \begin{itemize}
    	\item tutte le mappe di \(\lambda_0\) sono invertibili;
    	\item \(\forall r<s<t\)
 	      \[\Phi_{t\colon s} = \Lambda_{t\colon r} \circ \Lambda^{-1}_{s\colon r}\]
 	      è una mappa completamente positiva.
 \end{itemize}
\end{block}

\end{frame}

% Slide 3
\begin{frame}
\frametitle{Divisibilità oCP}
\begin{block}{CP divisibilità operativa [oCP]}
	\begin{columns}
		\begin{column}{0.5\textwidth}
			\begin{itemize}
				\item[r] preparazione stato \(\hat{\rho}_r\)
				\item[\(s_-\)] misurazione stato \(\hat{\rho}_{s_-}\)
				\item[\(s_+\)] preparazione stato \(\hat{\rho}_{s_+}\)
				\item[t] misurazione dello stato \(\hat{\rho}_t\)
			\end{itemize}
	    \end{column}
        \begin{column}{0.5\textwidth}
        	\resizebox{\textwidth}{!}{
\begin{tikzpicture}
\filldraw[rounded corners, fill=pastelblue] (-1,0) -- (1,0) -- (0,-1) -- cycle;
\filldraw[rounded corners, fill=pastelgreen] (-0.5,3) -- (-0.5,1) -- (-1.5,2) -- cycle;
\draw [rounded corners,line width=1mm, black] (0,0) -- (0,1) -- (1.5,1);
\draw [line width=1mm, black] (-0.5,2) -- (1.5,2);
\filldraw[rounded corners, fill=pastelred] (1.5,0.5) rectangle (3,2.5);
\draw [rounded corners,line width=1mm, black] (3,1) -- (3.5,1) -- (3.5,0.);
\draw [rounded corners,line width=1mm, black] (3.2,-0.3) -- (3.9,0.4);
\filldraw[rounded corners, fill=pastelblue] (4.5,0) -- (6.5,0) -- (5.5,-1) -- cycle;
\draw [rounded corners,line width=1mm, black] (5.5,0) -- (5.5,1) -- (7,1);
\draw [line width=1mm, black] (3,2) -- (7,2);
\filldraw[rounded corners, fill=pastelred] (7,0.5) rectangle (8.5,2.5);
\draw [line width=1mm, black] (8.5,2) -- (9,2);
\draw [rounded corners,line width=1mm, black] (8.7,1.7) -- (9.3,2.3);
\draw [line width=1mm, black] (8.5,1) -- (9.3,1);

\node[white] at (-0.9,2) {\LARGE \(\hat{\eta}_r\)};
\node[white] at (0,-0.4) {\LARGE \(\hat{\rho}_r\)};
\node at (3.55,-0.7) {\LARGE \(\hat{\rho}_{s_-}\)};
\node[white] at (5.5,-0.4) {\LARGE \(\hat{\rho}_{s_+}\)};
\node at (9.6,1) {\LARGE \(\hat{\rho}_{t}\)};

\node[white] at (2.3,1.5) {\LARGE \(\hat{U}_{s\colon r}\)};
\node[white] at (7.75,1.5) {\LARGE \(\hat{U}_{t\colon s}\)};

\end{tikzpicture}
}
        \end{column}	
    \end{columns}
	Canali accessibili sperimentalmente: \(\lambda := \{\Lambda_{s\colon r},\Lambda_{t\colon s}\}\). Il sistema è \emph{oCP} se:
	\begin{itemize}
		\item \(\forall r<s<t\)
		\[\Lambda_{t\colon r} = \Lambda_{t\colon s} \circ \Lambda_{s\colon r}\]
		%è una mappa completamente positiva.
	\end{itemize}
\end{block}
\pause
\begin{exampleblock}{Conditional non-signalling}
	L'oCP divisibilità richiede implicitamente che \(\Lambda_{t\colon s}\) non dipenda da \(\hat{\rho}_r\).
\end{exampleblock}
\end{frame}

% Slide 4
\begin{frame}
\frametitle{Relazioni tra classi di divisibilità}
\begin{columns}
	\begin{column}{0.68\textwidth}
		iCP \emph{non} è equivalente a oCP:
		\begin{itemize}
			\item<1-> \structure{\(\lambda_0\) non invertibile:} iCP non è ben definito.
			\item<2-> \structure{\(\lambda_0\) invertibile:} oCP\(\implies\)iCP, ma il viceversa non è vero.
		\end{itemize}
	\pause
	\emph{Conditional non-signalling} è necessario, ma non sufficiente alla oCP divisibilità.
	\end{column}
    \pause
	\begin{column}{0.30\textwidth}
		\begin{figure}
			\resizebox{\textwidth}{!}{\begin{tikzpicture}
	
	\fill[rounded corners,fill=pastelblue] (0,0) rectangle (6,4.5);
    \draw[color=black,dashed] (3,5.0) -- (3,-0.5);
    \fill[rounded corners,fill=pastelgreen] (0.5,0.25) rectangle (3,3.5);
    \fill[rounded corners,fill=pastelyellow] (1,0.35) rectangle (5.5,2.5);
    \fill[rounded corners,fill=pastelred] (1.5,0.45) rectangle (5.0,1.5);
    
    \node[white] at (3.25,1) {Markovianità};
    \node at (3.25,2) {oCP divisibilità};
    \node[white] at (1.75,3) {iCP divisibilità};
    \node[white] at (3,4) {Non divisibilità};
    \node at (1.5,4.8)  {\(\Lambda_{s:r}\) invertibile};
    \node at (4.5,4.8)  {\(\Lambda_{s:r}\) non invertibile};
\end{tikzpicture}}
		\end{figure}
	\end{column}
\end{columns}
\begin{columns}
	\begin{column}{0.3\textwidth}
		\structure{Markov  \(\not=\) oCP}\\
		oCP non  è comunque sufficiente ad escludere tutti gli effetti di memoria.
	\end{column}
    \pause
	\begin{column}{0.3\textwidth}
		\begin{figure}
			\resizebox{\textwidth}{!}{\input{img/ocp-no-markov.quantikz}}
		\end{figure}
	    \[\hat{U}_{s:r} = \hat{\mathcal{S}}_{S E_1},\, \hat{U}_{t:s} = \hat{\mathcal{S}}_{S E_2}\]
	\end{column}
    \pause
    \begin{column}{0.4\textwidth}
    	\[\Lambda_{s:r}\left[\hat{\rho}\right] = \hat{\eta}_1 \Tr{\hat{\rho}},\]
    	\[\Lambda_{t:s}\left[\hat{\rho}\right] = \hat{\eta}_2 \Tr{\hat{\rho}}  \]
    	\begin{itemize}[label={\checkmark}]
    		\item cond. non-sign.
    		\item oCP: \(\Lambda_{t\colon r} = \Lambda_{t\colon s} \circ \Lambda_{s\colon r}\)
    	\end{itemize}
	    \begin{itemize}[label={\(\times\)}]
	    	\item Markov: \(\hat{\rho}_{s_+t} \not= \hat{\rho}_{s_+} \otimes \hat{\rho}_{t}\)
	    \end{itemize}
    \end{column}
\end{columns}
\end{frame}

% Slide 5
\begin{frame}
\frametitle{Caratterizzazione della memoria oCP \& sviluppi ulteriori}
\begin{itemize}
\item<1-> L'oCP divisibilità rappresenta la \emph{markovianità in media}. 

\item<2-> La divisibilità non producono il corretto limite classico, quindi non hanno speranza di caratterizzare completamente la markovianità.
\begin{align*}
\text{iCP }&\left\{\mathbb{P}\left(X_{t},X_0\right)\right\}^n_{t\ge 1} \\
\text{oCP }&\left\{\mathbb{P}\left(X_{s},X_r\right)\right\}^n_{s>r \ge 0} \\
\text{markov }&\left\{\mathbb{P}\left(X_{n},\dots,X_0\right)\right\}
\end{align*}

\item<3-> \structure{Formalismo \emph{process tensor}} \cite{markovcondition}: generalizzazione della distribuzione di  probabilità.
\begin{itemize}
	\item markovianità quantistica tramite \emph{causal break}
	\item caratterizzazione dei processi oCP
\end{itemize}

\item<4-> \structure{\emph{Past-future correlation}} \cite{budini2018quantum}: definizione alternativa di markovianità quantistica, basata su una definizione classica alternativa
\[\text{markoviano}\implies C_{pf} := \sum \mathbb{P}(x,z|y)-\mathbb{P}(z|y)\mathbb{P}(x|y)=0\]
\end{itemize}
\end{frame}

% Slide Finale
\begin{frame}

\centerline{\huge Grazie per l'attenzione!} 

\vfill
\textbf{\structure{Bibliografia}}
\printbibliography
\centerline{\resizebox{2.3cm}{!}{\begin{tikzpicture}
  \bear[graduate=black,tassel=tasselred]
  \bearwear[shirt=bluunipi,
		    body deco={\node at (beartummy)
		    	         {\includegraphics[width=13.5pt]{img/unipi-noscritta-bianca.png}};}
           ]
\end{tikzpicture}}}

\end{frame}

\end{document}