\documentclass{beamer}
\usepackage{pack}

\usepackage[utf8]{inputenc}

\usetheme{PaloAlto}
\usecolortheme{seahorse} %seahorse, beaver


%Information to be included in the title page:
\title[CP doesn't mean Markov.]{\texttt{Inserisci titolo}}
\subtitle[it]{\texttt{Inserisci sottotitolo}}
\author[Luca Arnaboldi]{Luca Arnaboldi \\[1ex]
	{\small Relatore: Vittorio Giovannetti}}      
\institute[UniPi]{\includegraphics[height=1.45cm]{../img/unipi.png}}
\date[17-09-2020]{17 settembre 2020}
\logo{\includegraphics[height=1.45cm]{../img/unipi-noscritta.png}}

\begin{document}
	
\frame{\titlepage}

% Slide 1
\begin{frame}
	\frametitle[Slide 1]{Slide 1}
	Ora espongo un teomrema d'esempio \pause
	\begin{block}{Teorema}
		Uomo! Guarda che \(1+1 \not= 6\)
	\end{block}
\end{frame}

% Slide 2
\begin{frame}
\frametitle[Slide 2]{Slide 2}
Il mio libro prefrito è \cite{nielsen2010quantum}.
\end{frame}

% Slide 3
\begin{frame}
\frametitle[Slide 3]{Slide 3}
Il mio articolo preferito invece è \cite{CPdoesnotimply}
\end{frame}

% Slide 4
\section{Slides 4}
\begin{frame}
\frametitle{Slide 4}
\begin{equation}
\ket{\psi} = \frac{\ket{0}+\ket{1}}{\sqrt{2}}
\end{equation}
\end{frame}

% Slide 5
\begin{frame}
\frametitle[Slide 5]{Slide 5}
\end{frame}

% Slide Finale
\begin{frame}

\centerline{\huge Grazie per l'attenzione!}

\vfill
\textbf{\structure{Bibliografia}}
\printbibliography

\end{frame}

\end{document}